\documentclass[a4paper,12pt]{article}
\usepackage[english]{babel}
\usepackage[utf8]{inputenc}
\usepackage{amsmath}
\usepackage{graphicx}
\usepackage{caption}
\usepackage{subcaption}
\usepackage{epstopdf}
\usepackage{gensymb}
\usepackage{subcaption}
\usepackage{color}
\usepackage{listings}
\usepackage{setspace}
\usepackage{amssymb}
\usepackage{epstopdf}
\usepackage{enumitem}
\usepackage[parfill]{parskip}

\newcommand{\sr}{$^{88}$Sr$^+$}
%partial derivative
\newcommand{\pd}[2]{\frac{\partial #1}{\partial #2}}
%bra-ket notation
\newcommand{\bra}[1]{\left\langle #1 \right|}
\newcommand{\ket}[1]{\left| #1 \right\rangle}
\newcommand{\braket}[2]{\left\langle #1 \big| #2 \right\rangle}

%upright subscripts
\begingroup\lccode`~=`!
\lowercase{\endgroup\def~}#1{_{\mathrm{#1}}}
\mathchardef\exclam=\mathcode`!
\AtBeginDocument{\mathcode`!=\string"8000 }

%uprigth mu (or any greek letter)
\usepackage{xspace}
\def\mum{\ensuremath{\upmu \mathrm{m} \xspace}}
%\usepackage[]{mcode}
%\lstset{
%	literate = {ö}{{\"o}}1
%			{ä}{{\"a}}1
%			{€}{{\euro}}1
%} 
\title{Servo algorithm simulation for \sr ion clock}
\begin{document}
\maketitle

\section{Basic outline}

The aim of this document is to describe the servo simulator for a \sr ion clock.
The clock functions by matching a laser to a narrow transition of an \sr ion 
as accurately as possible.

We simulate the probing of a two-state system with a Rabi $\pi$-pulse.
The system is assumed to be in the ground state initially, so the probability
distribution is
\begin{equation}
    \rho_{ee}(f, \omega_0, \tau, S) = S \left( \frac{\pi}{2} \right)^2 \textrm{sinc}^2 
    \left( \frac{\sqrt{\pi^2 + (f-\omega_0)^2 \tau^2}}{2} \right) \text{,}
\end{equation}
where $f$ is the laser wavelength, $\omega_0$ the linecenter of the distribution,
$\tau$ the length of the $\pi$-pulse and $S$ the state prep factor.
If the ion is state-prepared i.e. we ensure that the ion is in the ground state
before the probe pulse, $S=1$; otherwise $S=1/2$.

Key assumptions:
\begin{itemize}
    \item The system is always in the ground state before the pulse.
    To simulate the lack of state preparation, the probability distribution
    is simply divided by 2.
    \item All laser pulses are $\pi$-pulses.
    \item The laser line is an ideal Dirac distribution.
    Thus, we know exactly where we are probing in the frequency space.
\end{itemize}

To simulate a clock cycle with laser at frequency $f$, we take the ideal excitation
probability $\rho_{ee}(f)$ and take $n$ draws from the binomial distribution
$B(n, \rho_{ee}(f))$.
This way, we can simulate the random shot noise from a cycle of multiple pulses.

In addition to shot noise, the model also includes laser and magnetic field
drift.
To complement these drifts, a servo algorithm is included to correct the
drift.
\subsection{Servo algorithm}

Naïvely, one would think that we could simply probe the ion at the transition line
center and attempt to maximize the amount of cycles where the ion is in the clock state.
However, a problem immediately presents itself: the probability of excitation
of a two-state system is symmetric as a function of detuning as seen in Fig. \ref{fig:exc_spec}.
If local oscillator does not maximize the transition probability, should it be 
tuned toward blue or red?
Additionally, due to small perturbations in environmental conditions, the laser
line and Zeeman peaks are constantly drifting and need periodic correction to remain at the reference
as precisely as possible.
To solve this problem, the ion is instead probed from both sides of the linecenter
at the assumed points where the spectrum reaches half maximum.
By sampling both of these multiple points many times, we can approximate the 
transition probabilities on the red and blue side, $p_R$ and $p_B$.
These can then be used to compose an algorithm for counteracting the detuning
by tuning the laser frequency using a servo \cite{Dube2015a}:
\begin{align}
  f_{i+1} &= f_{i} + E \\
  E &= G\: \frac{p_B - p_R}{p_B + p_R}
\end{align}
where $f_i$ is the frequency of the laser at time step $i$, $p_B$ and $p_R$ the 
blue- and red-detuned transition probabilities and $G$ the gain in Hz.

\begin{figure}
  \includegraphics[width=\textwidth]{../../thesis/figs/rabi_spec.pdf}
  \centering
  \caption{Excitation spectrum of an ideal two state system, along with
  the blue- and red-detuned sampling without and with offset.}
  \label{fig:exc_spec}
\end{figure}

To choose the optimal value for the gain $G$ such that it would always correct
the detuning in a single step, we must first revisit the definition of Allan
deviation, given by the square root of Eq. \ref{eq:AVAR}.
As we cannot directly measure the transition probability of at detuning $\delta$,
we instead perform many cycles and use the statistics to approximate the probability.
These measurements follow the binomial distribution, the variance of which is
$\sigma_{p_X} = p_X(1-p_X)$ where $p_X$ represents either $p_B$ or $p_R$.
Now we can redefine the Allan deviation as
\begin{equation}
  \sigma_y(\tau) = \frac{G}{\nu_0} \sqrt{\left(\frac{1-p_X}{p_X}\right) \frac{T_c}{\tau}}
  \label{eq:ADEV_gain}
\end{equation}
where $\nu_0$ is the transition frequency, $\tau$ the averaging time and $T_c$ the
cycle time of a single interrogation of the transition.
To arrive at the optimal $G$ value, we must minimize the residual servo tracking error.
We define $k_p$ as $k_p \pd{}{\delta} (p_B - p_R) |_{\delta=0}$, which we can numerically
evaluate by sampling $p_B$ and $p_R$ from the lineshape function.
Then $G= -2p_X/k_p$ and \eqref{eq:ADEV_gain} becomes
\begin{equation}
  \sigma_y(\tau) = \frac{-2 \sigma_{p_X}}{k_p \nu_0} \sqrt{\frac{T_c}{\tau}} \textrm{.}
\end{equation}

If state preparation is not possible, the average excitation probability at
the center of the transition line is only 50\%.
Due to the nature of binomial sampling, this increases the effect of shot noise
which in turn can cause erroneous servo corrections and technical shifts.
However, we can exploit the Zeeman effect of \sr ion to bypass this issue to 
achieve better statistics.
In bichromatic (BC) sampling a laser pulse at frequency $f_0$ is split in two
pulses which target different Zeeman peaks of a single Zeeman pair.
The splitting is done using an acousto-optic modulator.
Instead of one servo correcting the laser linecenter, we now have two.
One servo tracks the linear Zeeman splitting, $\Delta \nu_B$ and the other
tracks the frequency of the center of the peak pair $\Delta \nu_C$.
For a graphical representation, see Fig. \ref{fig:BC_example}

To control these servos, we interrogate the beaks on the blue and
red sides, similarly as in the single peak case.
Now that we have four such points, two for each peaks, we will use $+$ and $-$
to label which peak we are sampling.
$+$ corresponds to the peak with $m_J > 0$ and $-$ to the $m_J < 0$ one.
By splitting the pulse to two different frequencies, we are effectively
probing the distribution at two points simultaneously.
Thus, we can sum the transition probabilities of the two probing locations
to get new distributions with which we can control the servos.
The error signals associated with the magnetic field $E_B$ and center $E_C$ 
servos are
\begin{align}
  p!{outer} &= p_{R-} + p_{B+} \\
  p!{inner} &= p_{B-} + p_{R+} \\
  E_B &= G_B \frac{p!{outer} - p!{inner}}{p!{outer} + p!{inner}} \\
  p!{BB} &= p_{B-} + p_{B+} \\
  p!{RR} &= p_{R-} + p_{R+} \\
  E_C &= G_C \frac{p!{BB} - p!{RR}}{p!{BB} + p!{RR}} 
\end{align} 
The benefit of bichromatic sampling lies in the fact that $p!{outer}, p!{inner},
p!{BB}, p!{RR} \in [0,1]$.
Thus the error signals are less impacted by the random shot noise.
The servo gains $G_B$ and $G_C$ are calculated in a similar manner as the gain
for a single peak servo; only now $p_X$ represents one of the derived probabilities
and $k_p$ has to be redefined: $k_p = \pd{}{\delta} (p_X - p_Y) |_{\delta=0}$, where
$p_X$ and $p_Y$ correspond to the probabilities in the numerators of the
magnetic field and center error signals.

\begin{figure}
  \includegraphics[width=\textwidth]{../../thesis/figs/BC_sampling_example.pdf}
  \centering
  \caption{
    An example of bichromatic sampling.
    The two peaks of a Zeeman pair are shown with dashed lines, the red being
    the $m_J < 0$ and the blue the $m_J > 0$ peak.
    The crosses show the locations of the ideal sampling points for red- and
    blue-detuned samples for the servo algorithm.
    The line with endcaps shows the separation tracked by the magnetic field
    servo and the dotted line shows the frequency tracked by the center servo.
  }
  \label{fig:BC_example}
\end{figure}

The functions used in the module are self-documented.

\section{Example simulation workflow}

\begin{enumerate}
    \item The user provides general settings.
    These may include ambient magnetic field, magnetic field drift, 
    laser line drift etc.
    \item Using \texttt{sample\_initial\_values}, the finding and characterizing
    the transition lineshape is performed.
    The ideal lineshape is sampled at different points using binomial sampling.
    Afterwards, a lineshape function is fit to the sampled points using 
    \texttt{scipy.optimize}'s \texttt{curve\_fit}.
    \item The main simulation loop.
    For each main cycle step, the lineshape is probed numerous times to approximate
    transition probabilities at red and blue detuned points.
    While sampling, the possibly drifting laser line and magnetic field
    change values
    Then, using the gain setting calculated in the initialization stage, the 
    servos are corrected and the intermediary results saved.
\end{enumerate}
\bibliography{X:/BA-Projects/I/ITOC/JabRef/IonClock}
\bibliographystyle{unsrt}
\end{document}