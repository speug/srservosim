\documentclass[a4paper,12pt]{article}
\usepackage[english]{babel}
\usepackage[utf8]{inputenc}
\usepackage{amsmath}
\usepackage{graphicx}
\usepackage{caption}
\usepackage{subcaption}
\usepackage{epstopdf}
\usepackage{gensymb}
\usepackage{subcaption}
\usepackage{color}			
\usepackage{listings}
\usepackage{setspace}
\usepackage{amssymb}
\usepackage{epstopdf}
\usepackage{enumitem}
\usepackage[parfill]{parskip}

\newcommand{\sr}{$^{88}$Sr$^+$}
%partial derivative
\newcommand{\pd}[2]{\frac{\partial #1}{\partial #2}}
%bra-ket notation
\newcommand{\bra}[1]{\left\langle #1 \right|}
\newcommand{\ket}[1]{\left| #1 \right\rangle}
\newcommand{\braket}[2]{\left\langle #1 \big| #2 \right\rangle}

%upright subscripts
\begingroup\lccode`~=`!
\lowercase{\endgroup\def~}#1{_{\mathrm{#1}}}
\mathchardef\exclam=\mathcode`!
\AtBeginDocument{\mathcode`!=\string"8000 }

%uprigth mu (or any greek letter)
\usepackage{xspace}
\def\mum{\ensuremath{\upmu \mathrm{m} \xspace}}
%\usepackage[]{mcode}
%\lstset{
%	literate = {ö}{{\"o}}1
%			{ä}{{\"a}}1
%			{€}{{\euro}}1
%} 
\title{Servo algorithm simulation for \sr ion clock}
\begin{document}
\maketitle

\section{Basic outline}

The aim of this document is to describe the servo simulator for a \sr ion clock.
The clock functions by matching a laser to a narrow transition of an \sr ion 
as accurately as possible.

We simulate the probing of a two-state system with a Rabi $\pi$-pulse.
The system is assumed to be in the ground state initially, so the probability
distribution is
\begin{equation}
    \rho_{ee}(f, \omega_0, \tau, S) = S \left( \frac{\pi}{2} \right)^2 \textrm{sinc}^2 
    \left( \frac{\sqrt{\pi^2 + (f-\omega_0)^2 \tau^2}}{2} \right) \text{,}
\end{equation}
where $f$ is the laser wavelength, $\omega_0$ the linecenter of the distribution,
$\tau$ the length of the $\pi$-pulse and $S$ the state prep factor.
If the ion is state-prepared i.e. we ensure that the ion is in the ground state
before the probe pulse, $S=1$; otherwise $S=1/2$.

Key assumptions:
\begin{itemize}
    \item The system is always in the ground state before the pulse.
    To simulate the lack of state preparation, the probability distribution
    is simply divided by 2.
    \item All laser pulses are $\pi$-pulses.
    \item The laser line is an ideal Dirac distribution.
    Thus, we know exactly where we are probing in the frequency space.
\end{itemize}

To simulate a clock cycle with laser at frequency $f$, we take the ideal excitation
probability $\rho_{ee}(f)$ and take $n$ draws from the binomial distribution
$B(n, \rho_{ee}(f))$.
This way, we can simulate the random shot noise from a cycle of multiple pulses.

\section{Functions}
\subsection{Initialization}
These functions simulate setting the initial settings by sampling the lineshape.
\subsubsection{}

\section{Example simulation workflow}
\end{document}